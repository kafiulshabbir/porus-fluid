Simulation of two-phase flow in porous media has many applications in oil recovery, hydrology, electricity production, etc. Classical continuum models consider permeability to be a function of only saturation. Classical continuum models are unable to explain non-equilibrium effects. Advanced continuum models, such as the Kondaurov model considers permeability to be a function of a non-equilibrium parameter in addition to saturation. In order to better understand the non-equilibrium effects, it is necessary to develop non-continuum models at the scale of the pores, for example a network model. The objective of our research is to understand the physical meaning of the Kondaurov non-equilibrium parameter. In this article, we describe the network model we developed for simulating two-phase flow in porous media. At first we determine the pressures in each node by solving a system of linear equations. Then from the known pressures we calculate the flow rates. Lastly we decide an appropriate time step, distribute the fluids in the nodes, and perform displacement of the fluids in the tubes. Our model uses a novel method of distributing different phases in the nodes. We simulated the process of imbibition, where wetting fluid was located in the outer region with thicker radius, and non-wetting fluid in the inner region with thinner radius or finer pores. We measured the saturation of the wetting phase with respect to time in the region of finer pores. Our network model successfully shows that the wetting fluid invades the region of finer pores, and the saturation with respect to time rests to an equilibrium value. Also when produce a variation of radii in the inner region, we clearly observe that the capillary pressure increases with the decrease of wetting fluid in the inner region. 
