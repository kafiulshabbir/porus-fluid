\begin{enumerate}
	\item The novel method of distributing different fluids in the nodes, such the wetting fluid first goes into the tube with the thinner radius is valid, since it can model imbibition, where the saturation tends to an equilibrium value.
	
	\item The fact that the saturation $S(t)$ tends to an equilibrium value, implies that our network model can be used to simulate other relaxation phenomena, and it will help us better understand the physical meaning of the Kondaurov parameter $\xi$.
	
	\item The maximum number of connections a node in our network model can have is 4. Our algorithm can be easily extended to the more connects per node. Hence the same algorithm can be used to simulate a 3 dimensional network model.
	
	\item The fluctuations, once the saturation reaches an equilibrium value is very small. Also the error in terms of fluid lost in a closes system is extremely small. The errors were extremely small because Gaussian-elimination was used to find the pressure in each node, which is more accurate than iterative methods.	
\end{enumerate}


