Simulation of two-phase flow in porous media has applications in oil recovery, hydrology, electricity production where pressurized water is passed through heated pipes and is transformed into steam, etc. We developed a network model which uses a new method of distributing phases in the nodes, it can be used to simulate models containing about 1000 pores within a few minutes on a personal computer. The results explain relaxation phenomena, in the model of imbibition where the saturation of a phase was calculated with respect to time, the wetting fluid enters the region with thinner radius. The flow rate when a tube contains meniscus after necessary derivation is given by a modified Poiseuille’s equation. This produces n linear equations for a model with n nodes. The pressures at each node are the variables. Gaussian-elimination is used to solve the linear equations. Then the time step is decided and each meniscus in the tubes is displaced. The wetting fluid is distributed first in ascending order of the radius of the tubes. After this if there are more than two meniscus in a tube, the phases are combined such that the center of mass of the phases remain the same.  

